\section{Choices}

\subsection{Language} 
The language chosen for this project is TypeScript. 
Since the project is about the ecmascript ecosystem, it makes sense to use a language like Javascript. 
The Javascript ecosystem comes with a powerful package manager called npm. 
Npm hosts a library called espree which is a library created by the eslint team. 
This tool allows us to parse Javascript files and create tokenized versions of the scripts and abstract syntax trees. 

\noindent Typescript is a superset of Javascript and can use all the Javascript libraries. It also adds some features that make the development easier compared to vanilla Javascript.

\subsection{Frameworks}
The framework used in this project did not really matter, 
therefore I chose to use React.js. 
I chose to use Next.js as the main framework because it integrates webpack. 
It works great with TypeScript. 
It supports SCSS, which is far better than pure CSS. And it is built around React.js.
It also comes with a powerful static generation tool and integrates webpack natively.

\subsection{Desktop app}
Electron is the framework used to create a desktop app from the web app. 
It is used to provided the "import a folder" feature in the app. 
This feature could have been implemented using 
\href{https://developer.mozilla.org/en-US/docs/Web/API/HTMLInputElement/webkitdirectory}{this API}. 
However, this API is a non-standard API and should not be used in production. 
By using Electron, we can be assured that the user will be able to use this feature regardless of his platform and browser.



